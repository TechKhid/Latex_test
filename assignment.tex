\documentclass{article}
\usepackage{amsmath}
\usepackage{amstext}
\usepackage[utf8]{inputenc}
\usepackage{amsfonts}
\author{Priscilla Yaa Nyarkoa Oppong}
\date{}
\title{%
    Math446 Module Theory\\
    Assignment 1\\
    10714647}


\begin{document}
\maketitle
\noindent\ 
1. Let \emph{R} be a commutative unitary ring and let M be and R-module. Let $ r\in\emph{R}$
be fixed and let 


\[r\emph{M} = \{rx:x\in\emph{M}\}\ \ \textrm{and}\ \ \emph{M}r =\{x\in\emph{M}:rx = 0\}\]

(a) We want to show that  $r\emph{M}$ and $\emph{M}r$ are submodules of $\emph{M}$.
Considering r\emph{M}, since $\emph{M}$ is module, we know that $0\in\emph{M}$ and so 
for our fixed \emph{r}, we know that, $0 = \emph{r}0$ 
for $0\in\emph{M}$ and so  $0\in\emph{rM}$, hence  $\emph{rM}\ne\emptyset$\\
Now since $\emph{M}$ is an \emph{R}-module, we know that $\forall{x}\in\emph{M}$ and 
$\forall{r}\in\emph{M}$, $\emph{rx}\in\emph{M}$ and so by definition, $\emph{rM} \subseteq\emph{M}$. 
\\Choose $a, b\in\emph{rM}$ then $\exists{x}_1, x_2\in\emph{M}$ such that $a = rx_1, b = rx_2$.
Now choose some $s\in\emph{R}$. 
We want to show that $a + sb\in\emph{rM}$ \\

$$a+sb = rx_1 + s(rx_2) = rx_1 + (sr)x_2 = rx_1 + (rs)x_2 = rx_1 + r(sx_2)=\\
r(x_1 + sx_2)$$
We know that $x_1 + sx_2\in\emph{M}$ since $\emph{M}$ is a module and so 
$r(x_1 + sx_2)r\in\emph{M}$ so we have
that $r\in\emph{M}$ is a submodule of $\emph{M}$.\\
\\
Considering $\emph{M}r$, since $\emph{M}$ is a module, we know that $0\in\emph{M}$ 
and by the properties of 
the action of the ring $\emph{R}$ on the module $\emph{M}$, we know that 
$\forall{r}\in\emph{R}$, 
$r0 = 0$ and 
so $0\in\emph{M}r$
i.e $\emph{M}\ne\emptyset$. Now choose $x_1, x_2\in\emph{M}r$ and choose $s\in\emph{R}$. 
We want to show that 
$x_1 + sx_2\in\emph{M}r$. We know that $x_1 + sx_2\in\emph{M}$ since $\emph{M}$ is a module
$r(x_1 + sx_2) = r(x_1) + r(sx_2) = r(x_1) + rs(x_2) = 0 + s(rx_2) = 0 + 0 = 0$ 
since $ x_1, x_2\in\emph{M}r$ and 
$\emph{R}$ is a commutative ring. so we have that
$x_1 + sx_2\in\emph{M}r$ Hence $\emph{M}r$ is a  submodule of $\emph{M}$\\
(b)Let 
$\emph{R} = \mathbb{Z}, \emph{M} = \mathbb{Z}/\emph{n}\mathbb{Z}, \emph{n} = rs$
where $ra + sb = 1$ 
where $r\emph{M} = \emph{M}s$ 
i.e $\{rx : x\in\mathbb{Z}/\emph{n}\mathbb{Z}\} = \{x\in\mathbb{Z}/\emph{n}\mathbb{Z}: sx = 0\}$
\\
in other words $\forall{rx}\in{r}\emph{M}, s(rx) = O_m$ 
consider $rx\in{r}\emph{M}, rx\in\mathbb{Z}rs$
$s(rx) = sr(x)$ which is an $x$ multiple of $sr$ in $\mathbb{Z}/\emph{n}\mathbb{Z} = Om$
thus $r\emph{M}\subseteq\emph{M}s$. consider $y\in\emph{M}s$\\
we have that $sy = O_m$ then $y = rx, rx\in\mathbb{Z}/\emph{n}\mathbb{Z}$
\\
thus $\emph{M}s\subseteq{r}\emph{M}$\\
Hence $r\emph{M} = \emph{M}s$.

(c) Let $r\in\emph{R}$ be forced and consider the \emph{R}-module endomorphism\\
$c\sigma_r(m) = r\cdot{m}$.\\
\begin{itemize}  
\item ker$(\sigma_r) = \{m\in\emph{M} \vert  rm = 0\}$ which is exactly the definition
  of $\emph{M}_r$. Thus ker$(\sigma_r) = \emph{M}_r$.\\
\item $\emph{M}\vert\emph{M}_r =\{x + m_r \vert m_r\in\emph{M}_r\}$\\
\item $\sigma_r(m\vert{m_r}) = \{r(x+m_r)\vert {m_r}\in\emph{M}_r\}$\\
  $ = \{rx + rm_r \vert {m_r}\in\emph{M}_r\}$\\
  $ = \{rx\} = r\emph{M}$
  \\
  ker$(\sigma_r) = \{o_m\}\subseteq$, thus $(\sigma_r)$ is a bijection.\\\\
  \item $(\sigma_r)(m_1 + m_2) = r(m_1 + m_2)$\\
$= r(x_1 +mr+x_2+mr^2)$\\
$= r(x_1 +x_2+mr^1+mr^2)$\\
$= r(x_1 +mr^1)+r(x_2+mr^2)$\\
$= r(m_1)+r(m_2)$ where $m_1,m_2\in\emph{M}\vert\emph{M}_r$\\
thus $\sigma$ is a bicycle homomorphism from 
$\in\emph{M}\vert\emph{M}_r \longrightarrow r\emph{M}$ i.e\\
$\emph{M}\vert\emph{M}_r\cong r\emph{M}$.
\end{itemize}

(d) Let $\emph{M} = \emph{M}_1\bigoplus \emph{M}_2$ show that 
$r\emph{M} = r\emph{M}_1\bigoplus r\emph{M}_1$ and 
$\emph{M}_r = (\emph{M}_1)_r \bigoplus (\emph{M}_2)_r$ 
\\
Let $\emph{M} =  \emph{M}_1\bigoplus \emph{M}_2$ we know that $\emph{M}_1 \cap \emph{M}_2 = \{0\}$
and $\forall x \in \emph{M}, \exists, x_1, x_2 \in \emph{M}_1. \emph{M}_2$ respectively such that
$x = x_1+x_2$ thus 
$\forall rx\in r\emph{M}, rx = r(x_1 + x_2) = rx_1 + rx_2$ i.e 
$r\emph{M} \subseteq r\emph{M}_1 \bigoplus r\emph{M}_2$. Conversely suppose we have 
$rx_1+rx_2\in r\emph{M}_1 \bigoplus r\emph{M}_2$.\\
$rx_1 + rx_2 = r(x_1 + x_2)$\\$=r(x)$(since 
$\emph{M} = \emph{M}_1 \bigoplus \emph{M}_2$)\\Thus 
$r\emph{M}_1 \bigoplus r\emph{M}_2 \subseteq r\emph{M}$
\\
since $\emph{M} = \emph{M}_1 \bigoplus \emph{M}_2$\\
$\forall x \in\emph{M}_r \exists ! x_1, X_2 \in\emph{M}_1,\emph{M}_2 : x=x_1+x_2$\\
thus we know that $rx=r(x_1+x_2)$\\
$O=r(x_1+x_2)$.Since $\emph{M}_r$ is a submodule, we have 
$x=x_1+x_2\in\emph{M}_r\Leftrightarrow x_1,x_2\in\emph{M}_r$\\
Thus $rx_1, rx_2 = O$. Hence $\emph{M}_r =\emph{M}_1r + \emph{M}_2r $
\\\\
2. Let $\emph{R}$ be a ring\\
(a) Let $\emph{M}$ be an $\emph{R}$-module. Suppose $\emph{I}$ 
is a two sided Ideal of $\emph{R}$ with
the property that $\emph{I}\emph{M} = 0$. We want to show that the rule 
$$\bar{r} * m = rm + \emph{I}m$$\\
\\
gives a well defined action of $\emph{R}/\emph{I}$ on $\emph{M}$\\
We see that * : $\emph{R}/\emph{I}$ x $\emph{M}\longrightarrow\emph{M}$. 
Also, suppose that $\bar{r}_1 = \bar{r}_2$ and that $m_1 = m_2$ we want to show that 
$\bar{r}_1 * m_1 = \bar{r}_2 * m_2$\\
$\bar{r}_1 * m_1 = r_1m_1 + \emph{I}m_1 = r_1m_1 + 0 = r_1m_1$ 
but we know that $\bar{r}_1 = \bar{r}_2$ and 
$m_1 = m_2$ so 
$r_1m_1 = r_2m_2 = r_2m_2 + 0 = r_2m_2 + \emph{I}m_2 = \bar{r}_2 * m_2$\\
i.e. $\bar{r}_1 * m_2$. Hence we have that $*$ gives a well defined action of $\emph{R}/\emph{I}$ on
 $\emph{M}$.\\
We want to show that $\emph{M}$ is an $\emph{R}/\emph{I}$-module with the ring action $*$\\
We know that $\emph{M}$ is an $\emph{R}$-module and so ($\emph{M}$, $+$) is an abelian group.\\
We now show that the rest of the axioms of a module are satisfied.\\
Let  $\bar{r}_1, \bar{r}_2\in\emph{R}/\emph{I}$ and $m_1, m_2\in\emph{M}$\\
\\

\begin{itemize}
    \item $(\bar{r}_1 + \bar{r}_2) * m_1 = (r_1 + r_2)m_1 + \emph{I}m_1$\\
    $= r_1m_1 + r_2m_1 + \emph{I}m_1$\\
    $= r_1m_1 + \emph{I}m_1 + r_2m_1 + \emph{I}m_1$\\
    $= (\bar{r}_1 * m_1) + (\bar{r}_2 * m_1)$\\

    \item $(\bar{r}_1 \bar{r}_2) * m_1 = (\bar{r}_1 \bar{r}_2)m_1 + \emph{I}m_1$\\
    $= r_1(r_1m_1)+\emph{I}m_1$\\
    $= r_1((r_2m_1)+\emph{I}m_1) +\emph{I}m_1$\\
    $= \bar{r}_1 * (\bar{r}_1 * m_1)$\\

    \item $\bar{r}_1  * (m_1 + m_2) = r_1(m_1 + m_2) + \emph{I}(m_1 + m_2)$\\
    $= r_1m_1n + r_1m_2 + \emph{I}m_1 + \emph{I}m_2$\\
    $= r_1m_1n + \emph{I}m_1 + r_1m_2 + \emph{I}m_2$\\
    $= (\bar{r}_1 * m_1) + (\bar{r}_1 * m_2)$\\
\end{itemize}

Hence we have that $\emph{M}$ is an $\emph{R}/\emph{I}$ - module with the ring action $*$\\
(b)We want to show that every simple left $\emph{R}$-module 
is a cyclic left left $\emph{R}$-module\\
Let $\emph{M}$ be a simple module. Then, the only submodules of $\emph{M}$
 are the 0 submodule and $\emph{M}$\\
itself. Suppose also that $\emph{M}$ is not cyclic, that is $\emph{M}$ 
is not generated by each non zero element in $\emph{M}$
will be a submodule of $\emph{M}$ but this contradicts our assumption that the $\emph{M}$ is simple.
 Hence $\emph{M}$ must be generated by at least one of its 
elements and so $\emph{M}$ is cyclic.\\
(c)Consider $f:\emph{M}\to\emph{N}$ an \emph{R}-module homomorphism\\
$ker(\emph{f}) = \{m\in\emph{M}:f(m)=O_N\}$\\
$f(O_m)=O_N$ thus $O_m\in ker(f)$ and $m+rm_2\forall r\in\emph{R}$.\\
$f(m_1+m_2)=f(m_1)+rf(m_2)$\\
$O_N +O_N$
$=O_N$\\
Thus $m_1 + rm_2\in ker(f)$ and the submodule criterion is satisfied.\\
$lm(f)= \{n\in \emph{N}\vert\exists m\in\emph{M}, f(m)=n\}$\\
$O_N\in lm(f)$ since $f(O_m)=O_N$.Thus $lm(f)$ is not empty. Consider  $n_1,n_2\in lm(f).$
and $n_1 +n_2 \forall r\in\emph{R}$.\\
$f(m_1) +rf(m_2) = f(m)+f(rm_2) =f(m_1+m_2) = f(m_3)$\\
Thus $\exists m_3 = m_1+rm_2:f(m_3) =n_1+m_2$\\
$x n_1+rn_2\in lm(f)$ and the submodule criterion is satisfied.\\
\\
(d) Assume \emph{M} and \emph{N} are cyclic simple then\\
$ker(f) =\{O_m\}, \emph{M}$\\
$lm(f) = \{O_N\}, \emph{N}$.\\\\
if $ker(f) = \{O_m\}$ and $lm(f) = \{O_N\}$ \emph{f} is an isomorphism. 
if $ker(f) = \{O_m\}$ and $lm(f) = \emph{N}$ \emph{f} is an isomorph.\\
if $ker(f) = \emph{M}$ and $lm(f) = \{O_N\}$, \emph{f} is the zero map.\\
if $ker(f) = \emph{M}$ and $lm(f) = \emph{N}$ then \emph{f} is both injective and surjective so
\emph{f} is an isomorphism.
\\
\\
\\
\\
\\
\\
3.Suppose $\emph{M}$ is a finite abelian group. We know that then $\emph{M}$ 
is naturally a $\mathbb{Z}$-module.\\
\\
\underline{Claim}:\\
This action cannot be extended to make $\emph{M}$ into a $\mathbb{Q}$-module
\\
\\
\underline{Proof}:\\
Given any two unitary rings $\emph{M}$, $\mathbb{S}$ and an $\emph{R}$-module $\emph{M}$. If $\exists$
 a homomorphism $\emph{f}: \emph{S}\longrightarrow\emph{R}$ then \emph{M} is also an 
\emph{M}-module with the actionof \emph{S} on \emph{M} defined by 
$(\emph{s}, \emph{m})\longrightarrow\emph{f}\emph{m}$
for $\emph{s}\in\emph{S}$, $\emph{m}\in\emph{M}$.\\
But no such homomorphism exists for $\emph{S} = \mathbb{Q}$ and $\emph{R} = \mathbb{Z}$. 
Hence the action of $\mathbb{Z}$ on \emph{M} 
cannot be extended to make \emph{M} into a $\mathbb{Q}$-module.\\
\\
\\

4. Suppose that A $\le$ B, then a $\in$ A such that a $\in$ B and b $\in$ B (Since A $\subseteq$ B).\\
\begin{equation}
  Let\ A + C = B + C,\ then\ a + c \in A + C\ and\ b + c \in B + C\ for\ a\ \in A,\ b\ \in B\ and\ c \in C
\end{equation}
\begin{equation}Let\ A \cap C = B \cap C,\ then\ a = c\ for\ some\ a \in A\ and\ c \in C\ and\ b = c\ for\ some\ b \in B and\ c \in C.
\end{equation}
Note that A $\subseteq$ A + C and C $\subseteq$ A + C  ;\\B $\subseteq$ B + C and C $\subseteq$ B + C\\
so we have that A $\cap$ C $\subseteq$ A + C and B $\cap$ C $\subseteq$ B + C.\\
That is we have that a $\in$ A $\cap$ C $\subseteq$ A + C and b $\in$ B $\cap$ C $\subseteq$ B + C.\\
Hence from (2) and (3), we have that \\ 
a = b.\\
Also, from (2) and (3), we have that A $\cap$ C $\subseteq$ B $\cap$ C since A $\le$ B.\\ 
Thus a $\in$ A $\cap$ C such that a $\in$ B $\cap$ C.\\
So we can write that b = c = a\\
$\implies$ b = a.\\
therefore $A = B$.


-

\end{document}